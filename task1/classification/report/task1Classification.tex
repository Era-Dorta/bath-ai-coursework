\documentclass[12pt]{article}
 
\usepackage[margin=1in]{geometry} 
\usepackage{amsmath,amsthm,amssymb,bm}
 
\newcommand{\N}{\mathbb{N}}
\newcommand{\Z}{\mathbb{Z}}
 
\newenvironment{theorem}[2][Theorem]{\begin{trivlist}
\item[\hskip \labelsep {\bfseries #1}\hskip \labelsep {\bfseries #2.}]}{\end{trivlist}}
\newenvironment{lemma}[2][Lemma]{\begin{trivlist}
\item[\hskip \labelsep {\bfseries #1}\hskip \labelsep {\bfseries #2.}]}{\end{trivlist}}
\newenvironment{exercise}[2][Exercise]{\begin{trivlist}
\item[\hskip \labelsep {\bfseries #1}\hskip \labelsep {\bfseries #2.}]}{\end{trivlist}}
\newenvironment{reflection}[2][Reflection]{\begin{trivlist}
\item[\hskip \labelsep {\bfseries #1}\hskip \labelsep {\bfseries #2.}]}{\end{trivlist}}
\newenvironment{proposition}[2][Proposition]{\begin{trivlist}
\item[\hskip \labelsep {\bfseries #1}\hskip \labelsep {\bfseries #2.}]}{\end{trivlist}}
\newenvironment{corollary}[2][Corollary]{\begin{trivlist}
\item[\hskip \labelsep {\bfseries #1}\hskip \labelsep {\bfseries #2.}]}{\end{trivlist}}

\begin{document}
  
\title{Task 1.1. Supervised Learning: Standard Classifier}
\author{Garoe Dorta-Perez\\
CM50246: Machine Learning and AI}
 
\maketitle
 
\section{Introduction}
 
Given pictures from the world and been asked to classify them in several groups, we are faced with a problem of multi-class classification.
One of the options would be to create N one-against-all binary classifiers. Using $x$ to denote our data, $\omega$ to for the world state, and $lambda$ for the probability of observing the given class.

\begin{equation}
\label{bernoulliDistribution}
Pr(\omega|\mathbf{x}) = Bern_{w}[\lambda]\,
\end{equation}

However a better one involves using a categorical distribution to model our world. Where $\bm{\lambda}$ is a vector that contains a $\lambda$ for each class.

\begin{equation}
\label{categoricalDistribution}
Pr(\omega|\mathbf{x}) = Cat_{w}[\bm{\lambda}[\mathbf{x}]]\,
\end{equation}

\section{Mathematical derivation}

As stated in the introduction, we are going to fit a Categorical probability model into our data. 
Denoting $I$ as the total number of data points that we are given.  
Then, using Bayes' rule we have:

\begin{equation}
\label{bayes}
Pr(\theta | x_{1 \cdots I}) = \frac{\prod_{i = 1}^{I} Pr(\omega = k_{n} | x, \theta) Pr(\theta)} {Pr(x_{1 \cdots I})}\,
\end{equation}

We need N activation functions to enforce the constrains.
Since we are solving for multi-class classification a logistic sigmoid function as activation will not be valid.
Therefore a softmax function is used instead for each activation $a_{n}$.

\begin{equation}
\label{activations}
a_{n} = \phi_{n}^{T}x\,
\end{equation}

\begin{equation}
\label{softmax}
\lambda_{n} = softmax_{n}[a_{1}, a_{2} \cdots a_{N}] = 
\frac{exp[a_{n}]} {\sum_{m = 1}^{N} exp[a_{m}] }\,
\end{equation}

In order to simplify the calculations we are going to minimise the log of the probability:

\begin{equation}
\label{L}
\L = - log \sum_{i = 1}^{I} P(\omega = k | \mathbf{x}, \theta) = - log \sum_{i = 1}^{I} softmax[a_{n}]\,
\end{equation}
 
\section{Implementation}


\section{Results}

Digits dataset, with prior 100, initial phi ones
Elapsed time is 414.619122 seconds.
Hits: 71.87%

\section{Conclusion}


 
\end{document}