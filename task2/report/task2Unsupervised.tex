\documentclass[12pt]{article}
 
\usepackage[margin=1in]{geometry} 
\usepackage{amsmath,amsthm,amssymb,bm}
 
\newcommand{\N}{\mathbb{N}}
\newcommand{\Z}{\mathbb{Z}}
 
\newenvironment{theorem}[2][Theorem]{\begin{trivlist}
\item[\hskip \labelsep {\bfseries #1}\hskip \labelsep {\bfseries #2.}]}{\end{trivlist}}
\newenvironment{lemma}[2][Lemma]{\begin{trivlist}
\item[\hskip \labelsep {\bfseries #1}\hskip \labelsep {\bfseries #2.}]}{\end{trivlist}}
\newenvironment{exercise}[2][Exercise]{\begin{trivlist}
\item[\hskip \labelsep {\bfseries #1}\hskip \labelsep {\bfseries #2.}]}{\end{trivlist}}
\newenvironment{reflection}[2][Reflection]{\begin{trivlist}
\item[\hskip \labelsep {\bfseries #1}\hskip \labelsep {\bfseries #2.}]}{\end{trivlist}}
\newenvironment{proposition}[2][Proposition]{\begin{trivlist}
\item[\hskip \labelsep {\bfseries #1}\hskip \labelsep {\bfseries #2.}]}{\end{trivlist}}
\newenvironment{corollary}[2][Corollary]{\begin{trivlist}
\item[\hskip \labelsep {\bfseries #1}\hskip \labelsep {\bfseries #2.}]}{\end{trivlist}}

\begin{document}
  
\title{Task 1.1. Supervised Learning: Standard Classifier}
\author{Garoe Dorta-Perez\\
CM50246: Machine Learning and AI}
 
\maketitle
 
\section{Introduction}
 
We are given the task to classify some pictures of objects in several groups.
So, we are faced with a multi-class classification problem.
One of the options would be to create N one-against-all binary classifiers.
However an approach that naturally handles the multi-class nature of the problem is to use a categorical distribution to model our world. 

\section{The problem}

As stated in the introduction, we are going to fit a categorical probability model to our data.
Using Bayes' rule we have:

\begin{equation}
\label{bayes}
Pr(\theta | x_{1 \cdots I}) = \frac{\prod_{i = 1}^{I} Pr(\omega = k_{n} | x, \theta) Pr(\theta)} {Pr(x_{1 \cdots I})}\,
\end{equation}

Assuming the data can be classified using linear functions, N activations are needed to enforce the constrains.
Since we are solving for multi-class classification, a logistic sigmoid function as activation will not be valid.
Instead a softmax is used for each $a_{n}$.

\begin{align}
\label{activations}
a_{n} &= \phi_{n}^{T}x \\
\lambda_{n} = softmax_{n}[a_{1}, a_{2} \cdots a_{N}] &= 
\frac{exp[a_{n}]} {\sum_{m = 1}^{N} exp[a_{m}] }\,
\end{align}

For the prior we are going to use a normal distribution with zero mean and $\sigma^{2}$ variance.
In order to simplify the calculations we are going to minimise the log of the probability, where $y_{in}$ is the softmax expression for class $n$ and data $i$:

\begin{equation}
\label{L}
L = - log \sum_{i = 1}^{I} y_{in} + \frac{1}{2 \sigma^{2}} \phi^{T} \phi 
\end{equation}

We will use the gradient and Hessian updates shown below.

%Change for partial
\begin{align}
\label{Grad-Hess-Update}
\frac{\partial L}{ \partial \phi_{n}} &= \sum_{i = 1}^{I} \left( y_{in} - \delta \left[ \omega_{i} - n \right] \right) \mathbf{x}_{i}  + \frac{\phi} {\sigma^{2}} \nonumber \\
\frac{\partial^{2} L}{ \partial \phi_{m}\phi_{n}} &= \sum_{i = 1}^{I} y_{im} \left( \delta \left[ m - n \right] - y_{in} \mathbf{x}_{i} \mathbf{x}_{i}^{T}  \right) +  \frac{\delta \left[ m - n \right]}{\sigma^{2}} \nonumber \\
\end{align}

Predictions are calculated through Laplace approximation and Monte Carlo integration.

\begin{equation}
\label{Predictions}
predictions = \int y_{in} \mathcal{N}_{a} \left( \mu_{a}, \Sigma_{a} \right) da
\end{equation}
 
\section{Results}

We tested the classification algorithm with two different datasets.
The first one consists of hand drawn images of single digits from zero to nine, stored as a 784 vector of pixel values.
While the second one consists of pictures of eight different objects against a blue background, stored as a 576 vector of pixel values.
We present two tables showing prediction accuracy using different variances for the normal distribution prior and for several values of the $\phi_{0}$ vector. 

\begin{table}[h]
\begin{tabular}{l|lllllll}
Digits dataset &  &  &  &  &  &  &  \\ \hline
Prior Variance & \multicolumn{1}{l|}{1} & \multicolumn{1}{l|}{10} & \multicolumn{1}{l|}{100} & \multicolumn{1}{l|}{1000} & \multicolumn{1}{l|}{1} & \multicolumn{1}{l|}{1} & \multicolumn{1}{l|}{1} \\ \hline
$\phi_{0}$ & \multicolumn{1}{l|}{0.1} & \multicolumn{1}{l|}{0.1} & \multicolumn{1}{l|}{0.1} & \multicolumn{1}{l|}{0.1} & \multicolumn{1}{l|}{-1} & \multicolumn{1}{l|}{1} & \multicolumn{1}{l|}{2.5} \\ \hline
Prediction Accuracy & \multicolumn{1}{l|}{88\%} & \multicolumn{1}{l|}{86\%} & \multicolumn{1}{l|}{77\%} & \multicolumn{1}{l|}{48\%} & \multicolumn{1}{l|}{87\%} & \multicolumn{1}{l|}{87\%} & \multicolumn{1}{l|}{83\%} \\ \hline
\end{tabular}
\end{table}

\begin{table}[h]
\begin{tabular}{l|lllllll}
ETH-80-HoG dataset &  &  &  &  &  &  &  \\ \hline
Prior Variance & \multicolumn{1}{l|}{1} & \multicolumn{1}{l|}{10} & \multicolumn{1}{l|}{100} & \multicolumn{1}{l|}{1000} & \multicolumn{1}{l|}{1000} & \multicolumn{1}{l|}{1000} & \multicolumn{1}{l|}{1000} \\ \hline
$\phi_{0}$ & \multicolumn{1}{l|}{0.125} & \multicolumn{1}{l|}{0.125} & \multicolumn{1}{l|}{0.125} & \multicolumn{1}{l|}{0.125} & \multicolumn{1}{l|}{-10} & \multicolumn{1}{l|}{-1} & \multicolumn{1}{l|}{10} \\ \hline
Prediction Accuray & \multicolumn{1}{l|}{67\%} & \multicolumn{1}{l|}{74\%} & \multicolumn{1}{l|}{84\%} & \multicolumn{1}{l|}{89\%} & \multicolumn{1}{l|}{89\%} & \multicolumn{1}{l|}{89\%} & \multicolumn{1}{l|}{89\%} \\ \hline
\end{tabular}
\end{table}

For the \textit{Digits} dataset is clear that a smaller variance value, yields better predictions.
Thus indicating that a spiked normal distribution centred at zero is a reasonable guess.
With respect to the numerical minimization, we do not see significant improvements when using different $\phi_{0}$ values. 
Except for 2.5 which gives a decrease in performance, then indicating the existence of a worse local minima in that region.

%Check believe spelling, the same as the one above
While for the \textit{ETH-80} dataset, the trend is better accuracy as the variance increases.
Therefore the prior believe must be erroneous.
The effect of the initial guess for $\phi$ does not seem to affect the end result.
However, if set to zeros it does not converge or at least it does so quite slowly.
Which could be an indicator of a valley where the gradient is zigzagging.
 
\end{document}