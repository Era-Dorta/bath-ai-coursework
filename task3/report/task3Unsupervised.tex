\documentclass[12pt]{article}
 
\usepackage[margin=1in]{geometry} 
\usepackage{amsmath,amsthm,amssymb,bm}
 
\newcommand{\N}{\mathbb{N}}
\newcommand{\Z}{\mathbb{Z}}
 
\newenvironment{theorem}[2][Theorem]{\begin{trivlist}
\item[\hskip \labelsep {\bfseries #1}\hskip \labelsep {\bfseries #2.}]}{\end{trivlist}}
\newenvironment{lemma}[2][Lemma]{\begin{trivlist}
\item[\hskip \labelsep {\bfseries #1}\hskip \labelsep {\bfseries #2.}]}{\end{trivlist}}
\newenvironment{exercise}[2][Exercise]{\begin{trivlist}
\item[\hskip \labelsep {\bfseries #1}\hskip \labelsep {\bfseries #2.}]}{\end{trivlist}}
\newenvironment{reflection}[2][Reflection]{\begin{trivlist}
\item[\hskip \labelsep {\bfseries #1}\hskip \labelsep {\bfseries #2.}]}{\end{trivlist}}
\newenvironment{proposition}[2][Proposition]{\begin{trivlist}
\item[\hskip \labelsep {\bfseries #1}\hskip \labelsep {\bfseries #2.}]}{\end{trivlist}}
\newenvironment{corollary}[2][Corollary]{\begin{trivlist}
\item[\hskip \labelsep {\bfseries #1}\hskip \labelsep {\bfseries #2.}]}{\end{trivlist}}

\begin{document}
  
\title{Task 3. Relevance Vector Machine}
\author{Garoe Dorta-Perez\\
CM50246: Machine Learning and AI}
 
\maketitle
 
\section{Introduction}
 
SVM  are a quite popular choice in classification problems.
However they are inherently not probabilistic.
A technique that uses the same intuition is the Relevance Vector Machine.
It uses a full Bayesian approach with a prior that encourages sparseness.

\section{The problem}

First a dual model is used, previously we would model the data as a normal distribution.
Where $w$ is the world state, $x$ is the data point, $\phi$ are the paremeters of a linear function of the data and $\sigma$ is the standard deviation.
Using a transformation on the parameters as shown in Equation ~\ref{eq:normal}, so $\phi$ is now a weighted sum over the observed data points, where $\psi$ is a vector with the weights.
If we are working with few data of high dimensionality, this sum is faster to calculate, as it has less terms, that the original linear dependency.

\begin{equation}
\label{eq:normal}
Pr(w_i|\mathbf{x}_i) = Norm_{x_i}[\phi^T \mathbf{x}_i, \sigma^2],
\end{equation}

\begin{equation}
\label{eq:dual}
\phi = \mathbf{X} \psi,
\end{equation}

Using the dual parameters $\psi$, we encourage sparsernes in the model by using the prior defined in Equation \ref{eq:prior}.
Where $I$ is the number of data points, $Stud$ is a Student's t-distribution with $\nu$ degrees of freedom.
Using a product of t-distributions produces the desired sparseness since the areas with bigger probability density are in the origin and along the axis.

\begin{align}
\label{eq:prior}
Pr(\psi) &= \prod_{i = 1}^{I} Stud_{\psi_i} [0, 1, \nu],\\
&= \prod_{i=1}^I \frac{\Gamma \left( \frac{\nu + 1}{2} \right)}{ \sqrt{\nu \pi} \Gamma \left(\frac{\nu}{2} \right) } \left( 1 + \frac{\psi^2_i}{\nu} \right)^{- (\nu+1)/2},
\end{align}

A t-distribution is not conjugate to a Normal distribution, so there is no simple closed form solution for the posterior.
The solution will be to approximate the t-distributions by maximizing with respect to their hidden variables $h$, introduced in Equation \ref{eq:prHidden}, where $\mathbf{w}$ is the world state.
This leads to the marginal likelihood shown in Equation \ref{eq:likelihood}, where $\mathbf{I}$ is the identity matrix, $\mathbf{X}$ is a matrix with the data points and $\mathbf{H}$ is a matrix with all the hidden variables.

%Check this equation
\begin{equation}
\label{eq:prHidden}
Pr(\psi) = \prod_{i = 1}^{I} \int Norm_{\mathbf{w}}[0,1/h_i] Gam_{h_i}[\nu/2, \nu/2] dh_i,
\end{equation}

\begin{equation}
\label{eq:likelihood}
Pr(\mathbf{w}|\mathbf{X},\sigma^2) \approx \begin{array}{c}
		\\
      max \\
      \mathbf{ \mbox{\small H} }
    \end{array} \left[ Norm_{\mathbf{w}} \left[ \mathbf{0}, \mathbf{X}^T \mathbf{X} \mathbf{H}^{-1} \mathbf{X}^T \mathbf{X} + \sigma^2 \mathbf{I} \right] \prod_{d =1}^D Gam_{h_d} \left[ \nu /2, \nu/2 \right]  \right],
\end{equation}

The optimization is performed in three steps:
\begin{enumerate}
\item Optimize the marginal likelihood with respect to the hidden variables, using Equation \ref{eq:hiddenVar}.

\begin{equation}
\label{eq:hiddenVar}
h_i^{new} = \frac{1 - h_i \sum_{ii} + \nu}{\mu^2_i + \nu},
\end{equation}

\item Update $\boldsymbol{\mu}$ and $\mathbf{\Sigma}$, using Equation \ref{eq:muSigma}.

\begin{equation}
\label{eq:muSigma}
\begin{split}
\boldsymbol{\mu} &= \frac{1}{\sigma^2} \mathbf{A}^{-1} \mathbf{X}^T \mathbf{X} \mathbf{w},\\
\mathbf{\Sigma} &= \mathbf{A}^{-1},\\
\mathbf{A} &= \frac{1}{\sigma^2} \mathbf{X}^T \mathbf{X} \mathbf{X}^T \mathbf{X} + \mathbf{H},
\end{split}
\end{equation}

\item Optimize the marginal likelihood with respect to the variance parameter, using Equation \ref{eq:sigma}.
\end{enumerate}
 
\begin{equation}
\label{eq:sigma}
(\sigma^2)^{new} = \frac{1}{\mathbf{I} - \sum_i ( 1 - h_i \sum_{ii})}  \left( \mathbf{w} - \mathbf{X}^T \mathbf{X} \boldsymbol{\mu} \right)^T \left( \mathbf{w} - \mathbf{X}^T \mathbf{X} \boldsymbol{\mu} \right),
\end{equation}

After the training process, we only take the data points whose hidden variable $h_i$ is smaller than a threshold. Since a larger one means a small $\psi_i$, and hence no significance in the weighted sum.
 
\section{Results}

\end{document}